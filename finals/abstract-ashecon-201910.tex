\documentclass[12pt,]{article}
\usepackage{lmodern}
\usepackage{amssymb,amsmath}
\usepackage{ifxetex,ifluatex}
\usepackage{fixltx2e} % provides \textsubscript
\ifnum 0\ifxetex 1\fi\ifluatex 1\fi=0 % if pdftex
  \usepackage[T1]{fontenc}
  \usepackage[utf8]{inputenc}
\else % if luatex or xelatex
  \ifxetex
    \usepackage{mathspec}
  \else
    \usepackage{fontspec}
  \fi
  \defaultfontfeatures{Ligatures=TeX,Scale=MatchLowercase}
\fi
% use upquote if available, for straight quotes in verbatim environments
\IfFileExists{upquote.sty}{\usepackage{upquote}}{}
% use microtype if available
\IfFileExists{microtype.sty}{%
\usepackage{microtype}
\UseMicrotypeSet[protrusion]{basicmath} % disable protrusion for tt fonts
}{}
\usepackage[left=2cm,right=2cm,top=2cm,bottom=2cm]{geometry}
\usepackage{hyperref}
\hypersetup{unicode=true,
            pdftitle={Decision Support and Health Insurance Choice},
            pdfauthor={Ian McCarthy \& Evan Saltzman},
            pdfborder={0 0 0},
            breaklinks=true}
\urlstyle{same}  % don't use monospace font for urls
\usepackage{graphicx,grffile}
\makeatletter
\def\maxwidth{\ifdim\Gin@nat@width>\linewidth\linewidth\else\Gin@nat@width\fi}
\def\maxheight{\ifdim\Gin@nat@height>\textheight\textheight\else\Gin@nat@height\fi}
\makeatother
% Scale images if necessary, so that they will not overflow the page
% margins by default, and it is still possible to overwrite the defaults
% using explicit options in \includegraphics[width, height, ...]{}
\setkeys{Gin}{width=\maxwidth,height=\maxheight,keepaspectratio}
\IfFileExists{parskip.sty}{%
\usepackage{parskip}
}{% else
\setlength{\parindent}{0pt}
\setlength{\parskip}{6pt plus 2pt minus 1pt}
}
\setlength{\emergencystretch}{3em}  % prevent overfull lines
\providecommand{\tightlist}{%
  \setlength{\itemsep}{0pt}\setlength{\parskip}{0pt}}
\setcounter{secnumdepth}{0}

%%% Use protect on footnotes to avoid problems with footnotes in titles
\let\rmarkdownfootnote\footnote%
\def\footnote{\protect\rmarkdownfootnote}

%%% Change title format to be more compact
\usepackage{titling}

% Create subtitle command for use in maketitle
\providecommand{\subtitle}[1]{
  \posttitle{
    \begin{center}\large#1\end{center}
    }
}

\setlength{\droptitle}{-2em}

  \title{Decision Support and Health Insurance Choice}
    \pretitle{\vspace{\droptitle}\centering\huge}
  \posttitle{\par}
    \author{Ian McCarthy \& Evan Saltzman}
    \preauthor{\centering\large\emph}
  \postauthor{\par}
      \predate{\centering\large\emph}
  \postdate{\par}
    \date{October 2019}

\usepackage{setspace}
\doublespacing
\usepackage{titlesec}
\titlespacing{\section}{0pt}{12pt plus 2pt minus 1pt}{0pt plus 1pt minus 1pt}
\titlespacing{\subsection}{0pt}{12pt plus 2pt minus 1pt}{0pt plus 1pt minus 1pt}
\titlespacing{\subsubsection}{0pt}{12pt plus 2pt minus 1pt}{0pt plus 1pt minus 1pt}

\begin{document}
\maketitle

\hypertarget{background}{%
\subsubsection{Background:}\label{background}}

Health insurance markets are unique in many respects, not least of which
is the increasing complexity of choosing an optimal health insurance
plan. Such complexity has been well-documented in studies of health
insurance choice in the Medicare Advantage market (Abaluck and Gruber
2011; Ketcham et al. 2012; Gruber 2017). One way to reduce the burden of
this complexity is to provide professional decision support through
private insurance agents or public assistance programs, both of which
are available in the California health insurance exchanges created under
the Affordable Care Act (ACA). In this paper, we examine the role of
this decision assistance on health insurance plan choice and its
implications for consumer welfare.

\hypertarget{data}{%
\subsubsection{Data:}\label{data}}

Our final dataset consists of 3,200,080 household/year observations from
the California health insurance exhanges from 2014 to 2019. Our data
include a variety of household characteristics, plan choices and plan
characteristics, and information on the type of decision assistance used
by the enrollee (if any). From these data, we can identify each
household's set of possible health insurance plans, and we employ
premium and cost sharing subsidy formulas to calculate health insurance
costs for each possible plan for each household.

\hypertarget{methods}{%
\subsubsection{Methods:}\label{methods}}

We first estimate overall demand for health insurance using a nested
logit discrete choice model, as in Saltzman (2019). We then identify
specific instances in which the observed plan choice is dominated by
some other plan in an individual's choice set. For example, Gold and
Platinum tier plans are dominated for any household that is eligible for
cost-sharing subsidies and with incomes below 150\% of the federal
poverty level. We estimate the effect of decision assistance on
dominated choices with a linear probability model, allowing for year,
insurer, and region fixed effects.

\hypertarget{results}{%
\subsubsection{Results:}\label{results}}

Our nested logit models establish strong evidence that decision
assistance matters for insurance choice. Turning specifically to
dominated choices, our baseline results suggest that individuals with
decision assistance are 1.16 percentage points less likely to make a
dominated choice. On a base of 2.8\%, this reflects a 41\% decrease in
the probability of making a dominated choice.

We also find evidence fo heterogeneities in the effects of different
forms of decision assistance. In particular, we find that individuals
using publicly provided decision assistance are 0.91 percentage points
less likely to select a dominated plan, while individuals using a
private insurance agent are 1.26 percentage points less likely to select
a dominated plan.

\hypertarget{discussion}{%
\subsubsection{Discussion:}\label{discussion}}

Our current results provide strong evidence that decision assistance has
a significant and economically meaningful effect on health insurance
plan choice. In terms of dominated choices, our early results suggest
that this change in decision-making is welfare improving, and we
continue to explore other measures of choice dominance to further
confirm this result.

In future work, our identification strategy will exploit the Trump
administration's removal of cost-sharing subisidies from the exchanges
in 2018 and the subsequent response from insurers to increase premiums
on their silver plans (``silver loading''). This response significantly
changed the prevalence of dominated choices in each household's choice
set, and in this way, the removal of the cost-sharing subsidies offers
an exogenous change to the set of dominated plans available to each
household.

We are also extending our analysis of the differential effects between
public decision support versus assistance from private insurance
agents/brokers. This analysis will determine whether insurance brokers
are more likely to steer patients into plans offered by the sponsoring
insurer.

\hypertarget{references}{%
\section*{References}\label{references}}
\addcontentsline{toc}{section}{References}

\hypertarget{refs}{}
\leavevmode\hypertarget{ref-abaluck2011}{}%
Abaluck, Jason, and Jonathan Gruber. 2011. ``Heterogeneity in Choice
Inconsistencies Among the Elderly: Evidence from Prescription Drug Plan
Choice.'' \emph{American Economic Review} 101 (3): 377--81.

\leavevmode\hypertarget{ref-gruber2017}{}%
Gruber, Jonathan. 2017. ``Delivering Public Health Insurance Through
Private Plan Choice in the United States.'' \emph{Journal of Economic
Perspectives} 31 (4): 3--22.

\leavevmode\hypertarget{ref-ketcham2012}{}%
Ketcham, Jonathan D, Claudio Lucarelli, Eugenio J Miravete, and M
Christopher Roebuck. 2012. ``Sinking, Swimming, or Learning to Swim in
Medicare Part d.'' \emph{American Economic Review} 102 (6): 2639--73.

\leavevmode\hypertarget{ref-saltzman2019}{}%
Saltzman, Evan. 2019. ``Demand for Health Insurance: Evidence from the
California and Washington Aca Exchanges.'' \emph{Journal of Health
Economics} 63: 197--222.


\end{document}
