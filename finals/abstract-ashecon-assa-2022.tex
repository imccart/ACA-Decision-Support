\documentclass[12pt]{article}
\usepackage{graphicx,amssymb,amsmath,setspace,comment,verbatim,titling,pgf,lscape,color}
\usepackage[left=2cm,right=2cm,top=2.5cm,bottom=2cm]{geometry}
\usepackage[round]{natbib}
\usepackage{hyperref}
\usepackage{array}
\usepackage{bbm}
\usepackage{marginnote}
\usepackage[justification=centering]{caption}
%%\usepackage{breqn}
\newcommand{\pderiv}[2]{\frac{\partial#1}{\partial#2}}
%\usepackage{siunitx}
\newcolumntype{P}[1]{>{\raggedright\arraybackslash}p{#1}}
\hypersetup{colorlinks,%
						citecolor=black,%
						filecolor=black,%
						linkcolor=black,%
						urlcolor=blue,%
						}
\setstretch{1.5}
\setcounter{secnumdepth}{-\maxdimen} % remove section numbering

\setlength{\droptitle}{-50pt}

\title{Decision Assistance and Insurer Steering in Health Insurance}
\author{Ian McCarthy \& Evan Saltzman}
\date{March 2021}

\begin{document}
\maketitle

In some markets such as real estate, used automobiles, and insurance, consumers face significant cognitive challenges that may require the assistance of an agent or intermediary. The influence of agents on consumer decisions, and the firm's influence on the behaviors of its agents, have important implications for the efficiency of these markets. In this paper, we study the welfare effects of decision assistance and firm steering in health insurance. 

We use consumer-level enrollment data from the California Affordable Care Act (ACA) exchange, with 8.3 million household-year observations from 2014 to 2019. The data include a variety of household characteristics, plan choices and plan characteristics, and information on the type of decision assistance used by the enrollee (if any). Using these data, we can precisely identify each household's choice set, the premium paid for each plan in the choice set, and any premium and cost sharing subsidies for which the household is eligible.

Our initial analysis exploits the presence of "dominated plans" in the ACA exchanges, where we find that new enrollees using some form of decision assistance (agents or navigators) are approximately 28\% less likely to make a dominated choice. We then examine the causal effect of decision assistance on health insurance choice. Here, we find that decision assistance has a statistically significant and economically meaningful effect on plan choice, with enrollees 50\% more likely to select a silver plan when using some form of decision assistance and 20\% less likely to be uninsured.

We then turn to a structural analysis of insurer steering, commissions, and premiums. Preliminary results suggest that firms have considerable ability to steer consumers' decisions using agent commissions. For small firms, a \$1 increase in the plan commission has the same effect as a \$2 decrease in the plan premium on the probability of the plan being chosen. The effect is more pronounced for larger insurers, where a \$1 increase in the plan commission is equivalent to a \$3.50 decrease in the plan premium. 

Our results provide strong evidence that decision assistance has a significant and economically meaningful effect on health insurance plan choice. Decision assistance in general tends to improve plan choices, increasing the probability that enrollees select a silver plan and decreasing the probability of making a dominated choice. Gains to social welfare are potentially offset by insurer steering, which introduces the potential for alternative agent compensation schemes to improve welfare in this market.


\end{document}